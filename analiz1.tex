\hspace{1.25cm} Сайт (от англ. website: web — «паутина, сеть» и site — «место», буквально «место, сегмент, часть в сети») — система электронных документов (файлов данных и кода) частного лица или организации в компьютерной сети под общим адресом (доменным именем или IP-адресом).
Все сайты в совокупности составляют Всемирную паутину, где коммуникация (паутина) объединяет сегменты информации мирового сообщества в единое целое — базу данных и коммуникации планетарного масштаба. Для прямого доступа клиентов к сайтам на серверах был специально разработан протокол HTTP.
	Браузер —прикладное программное обеспечение для просмотра веб-страниц; содержания веб-документов, компьютерных файлов и их каталогов; управления веб-приложениями; а также для решения других задач. В глобальной сети браузеры используют для запроса, обработки, манипулирования и отображения содержания веб-сайтов. Многие современные браузеры также могут использоваться для обмена файлами с серверами ftp, а также для непосредственного просмотра содержания файлов многих графических форматов (gif, jpeg, png, svg), аудио-видео форматов (mp3, mpeg), текстовых форматов (pdf, djvu) и других файлов.
	Всемирная паутина  — распределённая система, предоставляющая доступ к связанным между собой документам, расположенным на различных компьютерах, подключенных к Интернету. Для обозначения Всемирной паутины также используют слово веб (англ. web «паутина») и аббревиатуру WWW. Всемирную паутину образуют сотни миллионов веб-серверов. Большинство ресурсов всемирной паутины основаны на технологии гипертекста. Гипертекстовые документы, размещаемые во Всемирной паутине, называются веб-стран	ицами. Несколько веб-страниц, объединённых общей темой, дизайном, а также связанных между собой ссылками и обычно находящихся на одном и том же веб-сервере, называются веб-сайтом. Для загрузки и просмотра веб-страниц используются специальные программы — браузеры (англ. browser)
 Разработанное веб-приложения предназначается для размещения информации о предоставляемых услугах, а так же размещения тематических статей.
Разработаны функции для отображения вышеперечисленной информации.
Веб-приложение не требует каких-либо входных данных. Обращение происходит путём ввода адреса сайта в адресную строку браузера.
Выходными данными являются html-страницы, которые преобразуются браузером в удобный для чтения пользователю вид.
Временные характеристики ограничиваются скоростью интернет соединения пользователя.
Надёжность веб-приложения находится на высоком уровне, так как пользователь не может вносить какие-либо данные или как-либо влиять на работу сайта.
Для разработки данного веб-приложения был выбран язык программирования php, так как он хорошо подходит для разработки шаблонизаторов. Также был выбран язык разметки HTML5, так как он широко используется для разработки веб-страница сайтов.