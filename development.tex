\subsection{\normalsize Описание используемых алгоритмов}
\label{sec:exp:pc}
\begin{center}
\textbf{Шаблонизатор} 
\end{center}
Концепция шаблонизатора состоит в том, чтобы разделять логику получения данных от логики отображения данных.

Шаблонизатор находит при помощи регулярных выражений и  записывает в массив строк, содержащий название tpl файлов, в файле страницы, затем извлекает содержимое этих файлов и подставляет их в готовую для вывода веб-страницу.
	Шаблонизатор реализован при помощи двух методов, таких как  getplaceholders, который находит названия tpl файлов и записывает их в массив и output, который распологает содержимое этих файлов в структуру страницы.
В файлах самих страниц, на который поступает запрос содержится лишь подключение шаблонизатора, создание экземпляра класса и вывод содержимого готовой страницы.
Реализация шаблонизатора:
\begin{verbatim}
<?php
class Template
{
	private $file;
	
	public function __construct($file)
	{
		$this->file = $file;
	}
	
	public function output()
	{

		$output = file_get_contents($this->file);
			$changed = true;
			while ($changed)
			{
				$changed = false;
				$placeholders = $this->get_placeholders();
				foreach ($placeholders as $key => $value)
				{
					$tag_to_replace = $value;
					$output_copy = $output;
					$template_name = 'templates/' . str_replace("{include(\"", "", $value);
					$template_name = str_replace("\")}", "", $template_name) . '.tpl';
					$output = str_replace($tag_to_replace, file_get_contents($template_name), $output);
					if ($output !== $output_copy)
					{
						$changed = true;
					}
				}
			}		
		return $output;
	}
	
	private function get_placeholders()
	{
		$template = file_get_contents($this->file);
		$pattern = '/{include\(.+?\)}/';
		
		$matches = array();
		preg_match_all($pattern, $template, $matches);
	
		return $matches[0];
	}

	
}	
\end{verbatim}	

Сами tpl файлы содержат html-код, при помощи которого конструируется будущий вид страницы.
В файлах страниц содержить подключение таблицы стилей, для более удобного и красочного отображения страницы, а так же файлы javascript, в которых реализован слайдер, используемый на двух страницах сайта, был использован слайдер easyslider.
Структура сайта состоит из заголовка, плавающего меню, блока с информационной частью.
В заголовке находится название страницы и ссылка на главную страницу сайта.
В плавающем меню расположены ссылки на остальные страницы сайта.
В информационном блоке содержится информация, зависящая от открытой пользователем страницы.
Реализация одного tpl файла:
\begin{verbatim}
	
			<div id="menu">
			<ul>
			<li><a class="links" href="contacts.php">Контакты</a></li>
			<li><a class="links" href="price.php">Услуги и цены</a></li>
			<li><a class="links" href="topics.php">Статьи</a></li>
			</ul>
			</div>
				<div id="content">
		
					<div id="container">

						<div id="sliderheader">
							<h1>Качественная проводка</h1>
						</div>
							<div id="slidercontent">
								<div id="slider">
									<ul>				
										<li><a href=""><img src="images/03.jpg" alt="Css Template Preview" /></a></li>
										<li><a href=""><img src="images/04.jpg" alt="Css Template Preview" /></a></li>
										<li><a href=""><img src="images/03.jpg" alt="Css Template Preview" /></a></li>
										<li><a href=""><img src="images/04.jpg" alt="Css Template Preview" /></a></li>
										<li><a href=""><img src="images/05.jpg" alt="Css Template Preview" /></a></li>			
									</ul>
								</div>
							</div>

					</div>

				</div>
\end{verbatim}
Реализация файла страницы:
\begin{verbatim}
<?php
include("template.php");
$template = new Template("templates/page-template.html");
echo $template->output();
\end{verbatim}