\subsection{\normalsize Описание используемых алгоритмов}
\label{sec:exp:pc}
\begin{center}
\textbf{Шаблонизатор} 
\end{center}
Концепция шаблонизатора состоит в том, чтобы разделять логику получения данных от логики отображения данных.

Шаблонизатор находит при помощи регулярных выражений и  записывает в массив строк, содержащий название tpl файлов, в файле страницы, затем извлекает содержимое этих файлов и подставляет их в готовую для вывода веб-страницу.
	Шаблонизатор реализован при помощи двух методов, таких как  getplaceholders, который находит названия tpl файлов и записывает их в массив и output, который распологает содержимое этих файлов в структуру страницы.
В файлах самих страниц, на который поступает запрос содержится лишь подключение шаблонизатора, создание экземпляра класса и вывод содержимого готовой страницы.	

Сами tpl файлы содержат html-код, при помощи которого конструируется будущий вид страницы.
В файлах страниц содержить подключение таблицы стилей, для более удобного и красочного отображения страницы, а так же файлы javascript, в которых реализован слайдер, используемый на двух страницах сайта, был использован слайдер easyslider.
Структура сайта состоит из заголовка, плавающего меню, блока с информационной частью.
В заголовке находится название страницы и ссылка на главную страницу сайта.
В плавающем меню расположены ссылки на остальные страницы сайта.
В информационном блоке содержится информация, зависящая от открытой пользователем страницы.